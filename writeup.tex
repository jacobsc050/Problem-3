% This is a simple sample document.  For more complicated documents take a look in the exercise tab. Note that everything that comes after a % symbol is treated as comment and ignored when the code is compiled.
\documentclass{article} % \documentclass{} is the first command in any LaTeX code.  It is used to define what kind of document you are creating such as an article or a book, and begins the document preamble

\usepackage{amsmath} % \usepackage is a command that allows you to add functionality to your LaTeX code

\title{Simple Sample} % Sets article title
\author{My Name} % Sets authors name
\date{\today} % Sets date for date compiled

% The preamble ends with the command \begin{document}
\begin{document} % All begin commands must be paired with an end command somewhere
\maketitle % creates title using information in preamble (title, author, date)

\section{Overview} % creates a section
%notice how the command will end at the first non-alphabet charecter such as the . after \LaTeX
The goal of this project is to numerically simulate the classical and quantum mechanical 
case of the CHSH correlator, and analyze the physical significance of each result. 
We do this by examining two different scenarios denoted as Problem 3.1 and Problem 3.2 in the 
following sections. The solution to both of these problems 
point to the same conclusion: quantum mechanics contains no hidden local variables that can account
for the measurment correlation between two entangled particles measured at a distance, 
in different lab setups. Consequently, we are forced to conclude that quantum mechanics is a 
fundementally non-local theory.

\section{Problem 3.1} 

Problem 3.1 presents our beloved scientists, Alice and Bob, measuring the angular momentum of classical particles. They set up a micro explosion that sends two particles to opposite sides of the lab, Alice recieves a fragment with angular momentum $J$, while Bob recieves another fragment with angular momentum $-J$. 

For each fragment, Alice will chose a direction $\alpha_{i}$ along which to measure her fragment. Bob will do the same, choosing an direction $\beta_{j}$. The final measurents for each measurement will be $\text{sign}[\alpha_{i} \cdot J]$ and $-\text{sign}[\beta_{j} \cdot J]$  respectively. For the first part of this problem, we calculate the CHSH correlator for the classical case of Alice and Bob's micro-exposion experiment. 

For the second part of this problem, we assume that Alice and Bob are measuring a singlet state of spin-1/2 particles. We 

\end{document} % This is the end of the document
